Open source implementation of Audio Video Transport Protocol (A\+V\+TP) specified in I\+E\+EE 1722-\/2016 spec.

Libavtp is under B\+SD License. For more information see L\+I\+C\+E\+N\+SE file.

\section*{Build}

Before building libavtp make sure you have all the required software installed in your system. Below are the requirements and their tested versions\+:


\begin{DoxyItemize}
\item Meson $>$= 0.\+43
\item Ninja $>$= 1.\+8.\+2
\end{DoxyItemize}

The first step to build libavtp is to generate the build system files.


\begin{DoxyCode}
1 $ meson build
\end{DoxyCode}


Then build libavtp by running the following command. The building artifacts will be created under the build/ in the top-\/level directory.


\begin{DoxyCode}
1 $ ninja -C build
\end{DoxyCode}


To install libavtp on your system run\+: 
\begin{DoxyCode}
1 $ sudo ninja -C build install
\end{DoxyCode}


\section*{A\+V\+TP Formats Support}

A\+V\+TP protocol defines several A\+V\+T\+P\+DU type formats (see Table 6 from I\+E\+EE 1722-\/2016 spec). Libavtp doesn\textquotesingle{}t support all of them yet. The list of supported formarts is\+:
\begin{DoxyItemize}
\item A\+AF (P\+CM encapsulation only)
\end{DoxyItemize}

\section*{Examples}

The {\ttfamily examples/} directory in the top-\/level directory provides example applications which demonstrate the libavtp functionalities. To build an example application run {\ttfamily \$ ninja -\/C build $<$example name$>$}.

Information about what exactly each example application does and how it works is provided in the beginning of the .c file from each application. 