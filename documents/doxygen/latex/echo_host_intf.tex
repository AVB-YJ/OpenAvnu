\section*{Description}

This interface module as a talker will push a configured string into the Media Queue for transmission. As a listener it will echo the received data to stdout. This is strictly for testing purposes and is generally intended to work with the \hyperlink{pipe_map}{Pipe mapping} module

~\newline
 \section*{Interface module configuration parameters}

\tabulinesep=1mm
\begin{longtabu} spread 0pt [c]{*2{|X[-1]}|}
\hline
\rowcolor{\tableheadbgcolor}{\bf Name }&{\bf Description  }\\\cline{1-2}
\endfirsthead
\hline
\endfoot
\hline
\rowcolor{\tableheadbgcolor}{\bf Name }&{\bf Description  }\\\cline{1-2}
\endhead
intf\+\_\+nv\+\_\+echo\+\_\+string &String that will be sent by the talker \\\cline{1-2}
intf\+\_\+nv\+\_\+echo\+\_\+string\+\_\+repeat&Number of copies of the string to send in each \textbackslash{} \\\cline{1-2}
\end{longtabu}
packet. ~\newline
 \textbackslash{} The repeat setting if used must come after the \textbackslash{} intf\+\_\+nv\+\_\+echo\+\_\+string in this file intf\+\_\+nv\+\_\+echo\+\_\+increment $\vert$ If set to 1 an incrementing number will be appended\textbackslash{} to the string intf\+\_\+nv\+\_\+tx\+\_\+local\+\_\+echo $\vert$ If set to 1 locally output the string to stdout \textbackslash{} at the talker intf\+\_\+nv\+\_\+echo\+\_\+no\+\_\+newline $\vert$ If set to 1 a newline will not be printed to the \textbackslash{} stdout intf\+\_\+nv\+\_\+ignore\+\_\+timestamp $\vert$ If set to 1 timestamps will be ignored during \textbackslash{} processing of frames. This also means stale (old) \textbackslash{} Media Queue items will not be purged. 